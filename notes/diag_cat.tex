\documentclass[a4paper,12pt]{article}
\usepackage[english]{babel} % for french
\usepackage[T1]{fontenc} % for accents
\usepackage[usestackEOL]{stackengine} % to be able to add the name of the teacher in the right corner
\usepackage{mathtools,amssymb, amsthm, stmaryrd, eufrak} % maths packages
\usepackage[backend=biber]{biblatex} % bibliography
\addbibresource{uni.bib} %bibliography file
\usepackage{graphicx, subfig} %for inserting figures
\usepackage{tikz, tikz-cd} %to make diagrams
\usepackage{hyperref}
\hypersetup{
    colorlinks=true,
    linkcolor=blue,
    filecolor=magenta,
    urlcolor=blue,
}
\usepackage{enumitem} %to enumerate stuff
\usepackage{minted}

\author{Raphaël Paegelow}
\title{diag :: a -> (a,a) : }
\begin{document}

\theoremstyle{definition}
\newtheorem{deff}{Définition}
\newtheorem{nota}{Notation}

\theoremstyle{remark}
\newtheorem{rmq}{Remarque}
\newtheorem{ex}{Exemple}

\theoremstyle{plain}

\newtheorem{thm}{Théorème}
\newtheorem{prop}{Proposition}
\newtheorem*{lemme}{Lemme}
\newtheorem{fait}{Fait}
\newtheorem*{conj}{Conjecture}


\maketitle
\begin{center}
\textbf{diag funtion\\
	Adjunctions and monads
}
\end{center}

First of all, here is a remainder of James implementation:


\begin{minted}[frame=single, escapeinside=||]{haskell}
diag :: a -> (a, a)
diag = join (,)

join :: Monad m => m (m a) -> m a

m = (e ->)
m = (^e)

join :: (e -> e -> a) -> e -> a
join f e = f e e

(,) :: a -> b -> (a, b)
\end{minted}
The main question here is how is that code related to the \textbf{Reader Monad} and can we interpret this code just as one of the two functors from the adjunction giving rise to this monad ?\\
Here is a remainder of what the \textbf{Reader Monad} is doing.\\
Let us fix $e \in \textbf{Set}$.
\begin{equation*}
\begin{split}
R_e:
\begin{array}{ccc}
Set & \to & Set\\
a & \mapsto & \{f: e \to a \}\\
\end{array}
\end{split}
\end{equation*}
With that in mind it is clear that $join: m(m x) \to m(x)$ is exactly the one associated to $m=R_e$. Now let's see which adjunction is hiding behind $R_e$.\\
\\
In fact we have to consider the slice category : $\textbf{Set/r}$ where $r \in \textbf{Set}$. This category has as objects all pairs of $(e,f)$ where $e \in \textbf{Set}$ and $f:e \to r$ and as morphisms functions from $e \to e'$ that commute with the functions to $r$:
\begin{center}
\begin{tikzcd}
e \ar[r, "g"] \ar[d, "f"'] & e' \ar[dl,"f'"]\\
r
\end{tikzcd}
\end{center}
Now that this is out of the way, here is the adjunction that gives rise to the \textbf{Reader Monad}:
\begin{center}
\begin{tikzcd}[column sep=large]
\textbf{Set} \ar[r,"L"', bend right=50] & \textbf{Set/r}  \ar[l, "R"', bend right=50]
\end{tikzcd}
\end{center}
where :
\begin{equation*}
\begin{split}
\begin{array}{ccc}
L:
\textbf{Set}  & \to & \textbf{Set/r} \\
a & \mapsto & ((a,r),p_2:(a,r) \to r)\\
\\
\\
R:
\textbf{Set/r}  & \to & \textbf{Set} \\
(a,f) & \mapsto & \{g:r \to a | f \circ g=id_r\}\\
\end{array}
\end{split}
\end{equation*}
Now there remains to see why $R_e=R \circ L$. Let's take $r=e$.\\ So consider $a \in \textbf{Set}$. $L(a)=((a,e),p_2)$.\\
Moreover, $R((a,e),p_2)=\{g:e \to (a,e) | p_2(g(x))=x, \forall x \in e\}$.\\ Such a $g=(g_1,g_2)$ and the condition is exactly that $g_2 = id_e$ meaning that only $g_1:e \to a$ is meaningful which is exactly what we wanted. There remains to check that it is behaving well on morphisms...\\
So here we have it (the idea is extracted from this \href{http://blog.higher-order.com/blog/2015/09/30/the-adjunction-that-induces-the-reader-monad/}{\textbf{blog post}}).\\
To conclude let's say something about the inital implementation.\\
$join$ function is in fact just $R\circ \epsilon \circ L$ where $\epsilon: L\circ R \to Id$ the counit of the adjunction. The global picture is the following:

\begin{center}
\begin{tikzcd}[column sep=large]
\textbf{Set $\times$ Set} \ar[r,"P"', bend right=50] & \textbf{Set} \ar[l, "\Delta"', bend right=50] \ar[r,"L"', bend right=50] & \textbf{Set/r}  \ar[l, "R"', bend right=50]
\end{tikzcd}
\end{center}
where $P$ is the product functor and $\Delta$ is the functor we are trying to understand.
According to the implementation let's understand why $\Delta=R\circ \epsilon \circ L \circ P$.
\end{document}
